\documentclass[12pt]{article}

\usepackage[dutch]{babel}
\usepackage{graphics}
\usepackage{graphicx}
\usepackage{amsmath}
\usepackage{eurosym}
\usepackage{mathtools}
\usepackage{amssymb}
\usepackage{enumitem}
\usepackage{tabto}
\NumTabs{10}

\begin{document}

\begin{titlepage}
\begin{center}

\textsc{\LARGE Universiteit Utrecht}\\[1.6cm]

\textsc{\Large Programmeren in de Wiskunde}\vspace{2pc}

{\huge \bfseries Verslag Eindopdracht}
\vspace{3pc}

\begin{minipage}{0.7\textwidth}
\begin{flushleft} \large
\emph{Begeleider:} \\
Emiel \textsc{Broeders} 
\end{flushleft}
\begin{flushleft} \large
\emph{Auteurs:}\\
Wessel \textsc{van Eeghen} \tab\#4007557\\
Mathijs \textsc{de Lepper}  \tab\#3987949\\
Jurriaan \textsc{Parie}  \tab\tab\#3938549
\end{flushleft}
\end{minipage}
\begin{minipage}{0.4\textwidth}
\end{minipage}
\vspace{10pc} 

\textsc{\Large Expressieboom}\\[4.0cm]
{\large \today}
\end{center}
\end{titlepage}

\newpage

\section{Inleiding}


\section{De Opdracht}
We hebben de aangeleverde code aanzienlijk uitgebreid. Naast de lijst met standaard functionaliteiten hebben we ook elementen toegevoegd uit de \textit{Extra}-lijst. Daarnaast hebben we nog een aantal (eigen) functies buiten de lijst om toegevoegd. 

\subsection*{Standaars Aanpassingen}

\subsection*{Extra Aanpassingen}
\begin{itemize}
\item \textbf{Het vertalen van een Expressie naar een string} \\
haakjes stukje?

\item \textbf{Gelijke bomen} \\
We willen twee bomen kunnen vergelijken. Dit doen we door in de verschillende klassen een overload van \texttt{def\_\_eq\_\_} te gebruiken. Op deze manier worden de twee expressiebomen vanaf een \textit{top-down} approach bekeken en vergeleken.
\begin{align*}
&\texttt{a = Expression.fromString('3sin(x)'))}\\
&\texttt{>>> print(a)}\\
&\texttt{3 * sin(x)}\\
&\texttt{b = Expression.fromString('sin(x)3')}\\
&\texttt{>>> print(b)}\\
&\texttt{sin(x) * 3}\\
&\texttt{>>> a == b}\\
&\texttt{False}
\end{align*}

\item \textbf{De klasse Variable} \\
Om ook variabelen te ondersteunen hebben we een klasse \textit{Variable} geschreven, als subklasse van \textit{Constant}. Hierin staat een constructor en een overload van  \texttt{def \_\_str\_\_}, die zichzelf als waarde teruggeeft en een overload van \texttt{def evaluate}. Als er een waarde is meegegeven voor de variabele, dan willen we deze waarde gebruiken in de berekening. Als dit niet het geval is, dan moet de variabele als variabele worden teruggegeven.
\begin{align*}
&\texttt{a = Expression.fromString('3sin(x)'))}\\
&\texttt{>>> a.evaluate()}\\
&\texttt{'3.0 * sin(x)'}\\
&\texttt{>>> a.evaluate({'x':math.pi/2})}\\
&\texttt{3.0}\\
\end{align*}


\item \textbf{Bekende en Onbekende Functies - de Unaire Boom} \\
Om niet alle elementen uit een berekening in een binaire boom te hoeven plaatsen, hebben we de unaire boom geintroduceerd. Dit biedt de mogelijkheid om ook functies in een berekening op te nemen. Deze unaire boom heeft slecht \'e\'en kind, hetgeen de functie evalueert. Dit kind kan zowel een binaire- danwel een unaire boom zijn. Zo is het mogelijk dat berekeningen met functies in functies ge\"evalueerd kunnen worden.
\begin{align*}
&\texttt{>>> Expression.fromString('sin x ** sinh log x + 3')}\\
&\texttt{sin(x) ** sinh(log(x)) + 3}
\end{align*}

\item \textbf{Negatie} \\
Het $-$ teken kent een dubbele betekenis. In het geval van \textit{negatie} vervangen we het $-$ teken in onze berekening door een $\sim$ om het vervolgens als een functie beschouwen. 
\begin{align*}
&\texttt{>>> expr =  print(Expression.fromString('-x / y + -sin -x'))}\\
&\texttt{-x / y + -sin(-x)}\\
%&\texttt{>>> print(expr.evaluate({'x':1.57, 'y':3.14}))}\\
%&\texttt{0.4999996829318346}
\end{align*}

\end{itemize}

\subsection*{Eigen Aanpassingen}


\newpage
\section{Taakverdeling}

\begin{center}
    \begin{tabular}{ | l | p{4cm}  | p{4cm} | p{4cm} |}
    \hline
    Dag & Jurriaan & Mathijs & Wessel \\ \hline
    11 juni & Carri\`eremiddag & Carri\`eremiddag & Code begrijpen \\ \hline
    16 juni & Presentatie + Code begrijpen & Presentatie + EvaluateFunction voor alles behalve variabelen & Nieuwe operatoren  $(-, *, /, **$ etc.$)$ \\ \hline
    18 juni & Begin variabelen in de expressieboom & EvaluateFunction afmaken & Reduceren van haakjes \\ \hline
    23 juni & Standaard en onbepaalde functies & Versimpeling en GitHub & Extra omschrijving + GitHub \\ \hline
    25 juni & Functies afmaken, begin negatie & \texttt{4log(sin(3xy))} is nu \texttt{4*log(sin(3*x*y))} & Versimpeling van de boom en versimpeling van de uitdrukkingen \\ \hline
    30 juni & Negatie is af en functies zonder haakjes & Bisection ge\"{i}mporteerd (werkt voor simpele functies) en PEP8 & Versimpelen is nu af \\ \hline
    3 juli & Verslag & Verslag & Verslag \\
    \hline
    \end{tabular}
\end{center}

\section{Discussie}
Waarom we ervoor gekozen hebben om bijvoorbeeld wel negatie te doen en niet differenti\"eren.


\section{Reflectie}

\begin{thebibliography}{9}
\bibitem{ftcs}
https://nl.wikipedia.org/wiki/Shunting-yardalgoritme
\end{thebibliography}


\end{document}